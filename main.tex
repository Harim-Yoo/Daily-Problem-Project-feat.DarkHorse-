\documentclass[12pt]{article}
\usepackage[margin=1in]{geometry}
\usepackage{amsmath, amssymb, amsthm}
\usepackage{comment} 
\newtheorem{problem}{Problem}

\newif\ifshowsolutions
\showsolutionsfalse

\ifshowsolutions
  \newenvironment{solution}{\par\noindent\textbf{Solution:}\itshape}{\par}
\else
  \excludecomment{solution}
\fi
  
\begin{document}
\begin{flushleft}
When I was in high school, I never thought about majoring in mathematics. I thought I was more of social science person, but the experience I had in college changed my thoughts. I still remember one incident that I wrote a totally wrong answer to one of the problem sets, but I received a fairly good mark because professor thought the approach to the solution looked novel.\\
\bigskip
Ever since then, I liked thinking deeply about a math problem. I stayed late night at Lunt Hall or McCormick building, where I slipped into an empty classroom and wrote down math problems on the blackboard, stared at them for a while to find solution without looking back at solution manual.\\
\bigskip
This passion led me to become a math instructor/problem-solving coach at ReachPrep in South Korea. To these days, I still solve high-school competition math problems. Some questions are still hard, and others are not. I am particularly interested in teaching fundamental concepts used in competition math to younger students.\\
\bigskip
In October 2024, I read ``Dark Horse" written by Todd Rose. This archive is a project based on ``Dark Horse'' to see how much I progress through with what I like. The level of difficulty for these problems is either early AIME or last seven problems in AMC 10 or 12.\\
\newpage


\section*{November 2024 Archive}

\input{problems/problem_2024-11-11.tex}
\begin{problem}
Given positive real numbers $a$, $b$, $c$, and $d$, if $(a+c)(b+d)=121$, determine the least value of $a+b+c+d$.
\end{problem}

\begin{solution}
According to AM-GM inequality, 
\begin{align*}
    \dfrac{(a+c)+(b+d)}{2} & \geq \sqrt{(a+c)(b+d)}\\
    \dfrac{a+b+c+d}{2} & \geq \sqrt{121}\\
    a+b+c+d &\geq 2(11)\\
    &\geq 22
\end{align*}
\end{solution}

\begin{problem}
Given three positive integers $a>b>c\geq1$, let $N$ be the number of triples $(a,b,c)$ such that $a+b+c=331$. Find the last three digits of $N$.
\end{problem}

\begin{solution}
Stars-and-bars imply that the total number of triples without condition equals $\binom{333}{2}$. Assume that $a=b>c$. Then, $2a+c=331$. Diophantine equation shows $166$ triples. Arrange $166$ triples by $\frac{3!}{2!}=3$ ways. Hence, subtract $498$ from $55278$ to retrieve $54780$. The answer is $780$.
\end{solution}

\begin{problem}
Compute the number of sequences that are arrangements of the numbers $1$, $2$, $3$, $4$, and $5$ and satisfy the requirement that no term in the sequence is larger than or equal to the square of the next term.
\end{problem}

\begin{solution}
First, $1$ should stay in the front of the sequence. Notice that $1<2^2$, $3<2^2$ and $4=2^2$. Hence, we only allow $12$ or $32$ as two consecutive terms in the sequence. Arrange $3$, $4$, and $5$, in $3!(=6)$ ways. Then, place $2$ next to either $1$ or $3$, in $2!(=2)$ ways. Therefore, there are $12$ sequences satisfying the given condition.
\end{solution}



\begin{problem}
Determine the number of ordered pairs of positive integers $(a,b)$ such that $a^b=3^{3^3}$.
\end{problem}

\begin{solution}
Since $3^{3^3}$ is $3^{27}$. there is one-to-one correspondence between the number of divisors of $27$ and $(a,b)$. Hence, there are $4$ pairs of $(a,b)$.
\end{solution}

\begin{problem}
Compute the least positive integer value of $n$ such that $10n$ and $12n$ have the same number of divisors.
\end{problem}

\begin{solution}
Let $n=2^a3^b5^c$. $10n$ and $12n$ have the equal number of divisors if and only if $(a+2)(b+1)(c+2)=(a+3)(b+2)(c+1)$. The smallest value of $n$ can be retrieved when $a=0$, $b=2$ and $c=0$. Thus, $n=9$.
\end{solution}

\begin{problem}
On the complex plane there are six complex numbers satisfying $z^6=1$. Given that $z$, $z^2$ and $z^3$ form a non-degenerate triangle, determine the ratio between the smallest area and largest area formed by $z$, $z^2$ and $z^3$.
\end{problem}

\begin{solution}
The smallest area is $\frac{\sqrt{3}}{4}$, whereas the largest area is $\frac{3\sqrt{3}}{4}$. Hence, the ratio equals $\frac{1}{3}$.
\end{solution}

\begin{problem}
Three customers in a grocery store saw there are five apples and four oranges. After they left, there was no lemon nor orange left. In how many ways would the customers buy these fruits such that each takes at least one fruit of ANY type?
\end{problem}

\begin{solution}
There are $315$ number of ways to distribute fruits to these customers. Apply De Morgan's Law and Principle of Inclusion and Exclusion. There are $87$ bad ways such that there is at least one person not buying any fruit. Hence, the answer equals $228(=315-87)$.
\end{solution}

\begin{problem}
Determine the sum of $x$-intercepts of a graph of $f(x)=(x-2)^7-x^7-4$.
\end{problem}

\begin{solution}
The graphs of two functions $y=(x-2)^7$ and $y=x^7+4$ intersect at two real points. Let $x=r$. then, $x=2-r$ is also the real solution. Hence, the sum of two real $x$-intercepts equals $2$. 
\end{solution}



\end{flushleft}
\end{document}
