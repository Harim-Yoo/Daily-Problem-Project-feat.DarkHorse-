\begin{problem}
Determine the number of ordered pairs of positive integers $(a,b)$ such that $a^b=3^{3^3}$.
\end{problem}

\begin{solution}
Since $3^{3^3}$ is $3^{27}$. there is one-to-one correspondence between the number of divisors of $27$ and $(a,b)$. Hence, there are $4$ pairs of $(a,b)$.
\end{solution}

\begin{problem}
Compute the least positive integer value of $n$ such that $10n$ and $12n$ have the same number of divisors.
\end{problem}

\begin{solution}
Let $n=2^a3^b5^c$. $10n$ and $12n$ have the equal number of divisors if and only if $(a+2)(b+1)(c+2)=(a+3)(b+2)(c+1)$. The smallest value of $n$ can be retrieved when $a=0$, $b=2$ and $c=0$. Thus, $n=9$.
\end{solution}

\begin{problem}
On the complex plane there are six complex numbers satisfying $z^6=1$. Given that $z$, $z^2$ and $z^3$ form a non-degenerate triangle, determine the ratio between the smallest area and largest area formed by $z$, $z^2$ and $z^3$.
\end{problem}

\begin{solution}
The smallest area is $\frac{\sqrt{3}}{4}$, whereas the largest area is $\frac{3\sqrt{3}}{4}$. Hence, the ratio equals $\frac{1}{3}$.
\end{solution}

\begin{problem}
Three customers in a grocery store saw there are five apples and four oranges. After they left, there was no lemon nor orange left. In how many ways would the customers buy these fruits such that each takes at least one fruit of ANY type?
\end{problem}

\begin{solution}
There are $315$ number of ways to distribute fruits to these customers. Apply De Morgan's Law and Principle of Inclusion and Exclusion. There are $87$ bad ways such that there is at least one person not buying any fruit. Hence, the answer equals $228(=315-87)$.
\end{solution}

\begin{problem}
Determine the sum of $x$-intercepts of a graph of $f(x)=(x-2)^7-x^7-4$.
\end{problem}

\begin{solution}
The graphs of two functions $y=(x-2)^7$ and $y=x^7+4$ intersect at two real points. Let $x=r$. then, $x=2-r$ is also the real solution. Hence, the sum of two real $x$-intercepts equals $2$. 
\end{solution}

