\begin{problem}
Given a circle centered at $O$, there exists an obtuse triangle $ABC$ inscribed in the circle such that $B$ is obtuse. If $AB=3$ and $BC=8$, and the height from $B$ to $\overline{AC}$ is $2$, find the area of the circle.
\end{problem}

\begin{solution}
Let $h$ be the length of the height from $B$ to $\overline{AC}$. Then, by similarity, $AB:h=2R:BC$. Hence, $AB(BC)=h(2R)$. Thus, $24=4R$. The radius is $4$, so the area of the circle equals $16\pi$.
\end{solution}

\begin{problem}
Determine the remainder when $3^{15}+5^{15}$ is divided by $17$. 
\end{problem}

\begin{solution}
According to Fermat's Little Theorem, $3^{16}\equiv 5^{16} \equiv 1 \pmod{17}$.
\begin{align*}
    3^{15}+5^{15} & \equiv \frac13+\frac15 \pmod{17}\\
    & \equiv \frac{18}{3}+\frac{35}{5} \pmod{17}\\
    & \equiv 6+7 \pmod{17}\\
    & \equiv 13 \pmod{17}
\end{align*}
\end{solution}

\begin{problem}
Given positive real numbers $a$, $b$, $c$, and $d$, if $(a+c)(b+d)=121$, determine the least value of $a+b+c+d$.
\end{problem}

\begin{solution}
According to AM-GM inequality, 
\begin{align*}
    \dfrac{(a+c)+(b+d)}{2} & \geq \sqrt{(a+c)(b+d)}\\
    \dfrac{a+b+c+d}{2} & \geq \sqrt{121}\\
    a+b+c+d &\geq 2(11)\\
    &\geq 22
\end{align*}
\end{solution}

\begin{problem}
Given three positive integers $a>b>c\geq1$, let $N$ be the number of triples $(a,b,c)$ such that $a+b+c=331$. Find the last three digits of $N$.
\end{problem}

\begin{solution}
Stars-and-bars imply that the total number of triples without condition equals $\binom{333}{2}$. Assume that $a=b>c$. Then, $2a+c=331$. Diophantine equation shows $166$ triples. Arrange $166$ triples by $\frac{3!}{2!}=3$ ways. Hence, subtract $498$ from $55278$ to retrieve $54780$. The answer is $780$.
\end{solution}

\begin{problem}
Compute the number of sequences that are arrangements of the numbers $1$, $2$, $3$, $4$, and $5$ and satisfy the requirement that no term in the sequence is larger than or equal to the square of the next term.
\end{problem}

\begin{solution}
First, $1$ should stay in the front of the sequence. Notice that $1<2^2$, $3<2^2$ and $4=2^2$. Hence, we only allow $12$ or $32$ as two consecutive terms in the sequence. Arrange $3$, $4$, and $5$, in $3!(=6)$ ways. Then, place $2$ next to either $1$ or $3$, in $2!(=2)$ ways. Therefore, there are $12$ sequences satisfying the given condition.
\end{solution}

\begin{problem}
Determine the number of ordered pairs of positive integers $(a,b)$ such that $a^b=3^{3^3}$.
\end{problem}

\begin{solution}
Since $3^{3^3}$ is $3^{27}$. there is one-to-one correspondence between the number of divisors of $27$ and $(a,b)$. Hence, there are $4$ pairs of $(a,b)$.
\end{solution}

\begin{problem}
Compute the least positive integer value of $n$ such that $10n$ and $12n$ have the same number of divisors.
\end{problem}

\begin{solution}
Let $n=2^a3^b5^c$. $10n$ and $12n$ have the equal number of divisors if and only if $(a+2)(b+1)(c+2)=(a+3)(b+2)(c+1)$. The smallest value of $n$ can be retrieved when $a=0$, $b=2$ and $c=0$. Thus, $n=9$.
\end{solution}

\begin{problem}
On the complex plane there are six complex numbers satisfying $z^6=1$. Given that $z$, $z^2$ and $z^3$ form a non-degenerate triangle, determine the ratio between the smallest area and largest area formed by $z$, $z^2$ and $z^3$.
\end{problem}

\begin{solution}
The smallest area is $\frac{\sqrt{3}}{4}$, whereas the largest area is $\frac{3\sqrt{3}}{4}$. Hence, the ratio equals $\frac{1}{3}$.
\end{solution}

\begin{problem}
Three customers in a grocery store saw there are five apples and four oranges. After they left, there was no lemon nor orange left. In how many ways would the customers buy these fruits such that each takes at least one fruit of ANY type?
\end{problem}

\begin{solution}
There are $315$ number of ways to distribute fruits to these customers. Apply De Morgan's Law and Principle of Inclusion and Exclusion. There are $87$ bad ways such that there is at least one person not buying any fruit. Hence, the answer equals $228(=315-87)$.
\end{solution}

\begin{problem}
Determine the sum of $x$-intercepts of a graph of $f(x)=(x-2)^7-x^7-4$.
\end{problem}

\begin{solution}
The graphs of two functions $y=(x-2)^7$ and $y=x^7+4$ intersect at two real points. Let $x=r$. then, $x=2-r$ is also the real solution. Hence, the sum of two real $x$-intercepts equals $2$. 
\end{solution}

\begin{problem}
Suppose $f(x)=x^3-x+1$ has roots $r$, $s$, and $t$. Determine $\frac{r^3}{r-1}+\frac{s^3}{s-1}+\frac{t^3}{t-1}$.
\end{problem}

\begin{solution}
According to Viete's formula, $f(r)=f(s)=f(t)=0$. Hence, $r^3=r-1$, $t^3=t-1$ and $s^3=s-1$. Hence, the answer is $3$.
\end{solution}

\begin{problem}
Determine the maximum possible value of $xy$ if $x^2+5xy+y^2=1001$ for real $x$ and $y$.
\end{problem}

\begin{solution}
According to AM-GM inequality, $x^2+5xy+y^2\geq 7xy$. Hence, $xy\leq 143$. Check that $x=y=\sqrt{143}$.
\end{solution}

\begin{problem}
Determine the sum $\sum_{k=2}^{100}\frac{1}{\binom{k}{2}}$.
\end{problem}

\begin{solution}
This telescopes so the sum changes into $2(1-\frac{1}{101})=\frac{200}{101}$.
\end{solution}

\begin{problem}
There are three rows with three seats each. If Mary, May and Mo wish to sit down, in how many ways can they be seated in different rows and different columns?
\end{problem}

\begin{solution}
Send Mary, May, and Mo, in that order. The answer is $9\times4\times1=36$.
\end{solution}

\begin{problem}
In how many ways can four $($'s and $)$'s be arranged in a line such that $)$ must be written after $($?
\end{problem}

\begin{solution}
This is a typical Catalan number problem for $4\times4$. Hence, there are $\frac{1}{5}\binom{8}{4}=14$.
\end{solution}

\begin{problem}
Points $P$ and $Q$ lie outside and inside square $AIME$, respectively such that $\triangle API \cong \triangle MQI$ and $m\angle API = 90^\circ$. Suppose that $AP=1$ and $PQ=24$. Compute $[AIME]$ where $[AIME]$ refers to the area of $\square AIME$.
\end{problem}

\begin{solution}
Apply Pythagorean Theorem to conclude that $PI=12\sqrt{2}$ and $AP=1$, so $AI=17$. Hence, $AI^2=289=[AIME]$.
\end{solution}

\begin{problem}
A polynomial equation $x^3+px^2+12x+q=0$ has integer roots, one of which is $1$. Find the maximal possible value of $p$.
\end{problem}

\begin{solution}
According to Viete's formula, let $x^3+px^2+12x+q=(x-1)(x-a)(x-b)$ for two other solutions $a$ and $b$. Then, $(a+1)(b+1)=13$ implies that $(a,b)=(12,0)$, $(0,12)$, $(-14,-2)$ and $(-2,-14)$. Hence, $p=-a-b-1$ is either $15$ or $-13$. The largest possible value is $15$.
\end{solution}

\begin{problem}
Let $n$ be the number of degree $5$ polynomials $f(x)$ with non-negative integer coefficients such that $f(1)=10$ and $f(-1)=4$. Find the total possible number of $f(x)$s.
\end{problem}

\begin{solution}
Apply stars and bars twice to get $216(=36\times6)$ polynomials of $x$.
\end{solution}